\chapter{METODOLOGIA DA REVISÃO SISTEMÁTICA \label{appendix:revisao-literatura}}

Uma busca sistemática da literatura foi realizada e os seguintes parâmetros se destacam:
\begin{itemize}
   \item \textbf{Query de busca}:  \enquote{cid E aprendizado E maquina}, \enquote{codigo E icd}, \enquote{ehr E classification}, \enquote{ehr E ensemble E learning E word2vec E fasttext}, \enquote{ehr E mimic E embedding E similarity}, \enquote{ehr E word2vec}, \enquote{icd E bert}, \enquote{icd E classification}, \enquote{icd E deep E learning}, \enquote{mimic E ensemble E learning}, \enquote{mimic E lstm}, \enquote{mimic E rnn}, \enquote{mimic E transformer}, \enquote{mimic-III E word2vec E similarity}, e \enquote{subwording E tokenization};
   \item \textbf{Motor de busca}: \textit{google.com.br}, \textit{scholar.google.com.br}, e \textit{semanticscholar.org};
   \item \textbf{Variável de exclusão}: ano de publicação menor que 2016.
\end{itemize}
 
 Para cada documento encontrado, a seguinte tabela foi preenchida:
 \begin{itemize}
   \item Título do artigo;
   \item Data da publicação;
   \item O objetivo principal do documento;
 \end{itemize}
 
 Quando o objetivo principal do documento encontrado se relacionava com o objetivo principal desse trabalho, os seguintes campos eram preenchidos:
 \begin{itemize}
   \item A necessidade da pesquisa;
   \item A metodologia apresentada;
   \item O resultado final e trabalhos futuros;
  \end{itemize}
  
Ao final dessa etapa, 66 documentos constavam na tabela. Após uma análise das informações, 29 documentos foram selecionados como relevantes. Um estudo detalhado desses documentos foi realizado e o processo de \textit{snowballing} foi empregado para identificar documentos adicionais relevantes para compor as referências.

Além disso, referências usualmente utilizadas na área de \gls{pln}, notoriamente as quais possuem um elevado número de citações, não passaram pelo processo descrito anteriormente.